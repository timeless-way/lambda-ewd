\documentclass[11pt, a4paper]{article}
\usepackage[english]{babel}
\usepackage{fancyvrb}
\usepackage{amsmath, amssymb}
\usepackage[euler-digits,euler-hat-accent]{eulervm}

\usepackage{newpxtext}

\begin{document}

\title{Notes (on Structured Programming?)}
\author{Huub van Thienen}
\maketitle

\section{Design principles}

\begin{enumerate}
\item Some design principles in random order.

\item Every 10 years I design a new programming language and its compiler/interpreter.

\begin{list}{}{}
\item 1984: Portable Logo
\item 1994: First incarnation of Prospero
\item 2007: Remake of Prospero in Java, using OO implemetation techniques
\item 2018: EWD
\end{list}

\item After its creation in 2007, java-prospero has been substantially extended and 
redesigned. It now contains many innovative programming concepts but tends to become 
a bit overloaded.

\item Design of Prospero always focussed on concepts and never on efficient implementation.
This proved to be a fruitful approach.

\item Many things were learned from Prospero but the sheer size of the language and 
the remnants of not-so-successful experiments clutter up. It is time to streamline the 
ideas and try to collect them into a new consistent language.

\item During the past decade two developments were made that influence our choices of 
language technology:

\begin{list}{}{}
\item A: Python went from obsure scripting language to a mature mainstream language

\item B: In the field of compiler construction packrat parsing and PEG grammars 
became dominant.
\end{list}

\item During development of Prospero 2 the focus has always been much more on language 
design and implementation and much less on using the language (and not at all on 
documenting the language). This will be the case during a new linguistic project as well. For that reason, there will be no attempt to produce a stand alone version of ewd, rather, only a few components will be provided that can be used in an iPython terminal or a Jupyter Notebook

\item It must be really, truly, absolutely lazy! Perhaps reading Simon Peyton Jones 
stuff will help with implementing a lazy evaluator.

\item The first attempt is based on an implementation of the Core Language of Lester and Peyton Jones.

\item While implementing the language, care must be taken not to modify the syntax with the sole purpose of simplifying the implementation.
On the contrary, as the whole exercise is done only to learn about implementations, simplification will be counter-productive.

\end{enumerate}

\section{Implementation notes}

\paragraph{2018-03-03T12:22}

A \emph{Program} consists of a source text and a Language in which the source is written. 
The source text is usually supplied as a file, but it is entirely read in memory before any processing begins.
A \emph{Language} consists of its syntax and semantics. (Currently there is no semantics yet, working on that). 
The syntax is provided as a parser instance.
This instance contains a method parse that creates an AST for the source.

Parser instances are created using tatsu (FKA grako) in one of two possible ways. 
Both ways use the same syntax specification ewd.tatsu. 
The method \verb|Language.__init__| is adapted to both techniques.

The first technique creates a tatsu model in runtime. 
This is very useful for quick experiments with the grammar. 
Create a language using

\verb|ewd = Language(grammar_file='ewd')|
The Language constructor will generate a tatsu model that is subsequently used to parse sources.

The second technique offers more flexibility at the cost of a more complex build structure.
First, generate a python module for the grammar:\\
\verb|hvtools.generate_python_module('ewd');|
Import this module in \verb|ewd.py| (the module is called \verb|ewd_parser)|. 
\verb|ewd_parser| contains a class \verb|EwdParser|. 
Create an instance of this class and pass it to the Language constructor:
\verb|parser = ewd_parser.EwdParser()|
Pass this instance to the Language constructor:
\verb|ewd = Language(parser_instance = parser)|

\paragraph{NOTE:} \verb|Language(...)| has two paramaters of which EXACTLY ONE must be specified: either use
\verb|Language(parser_instance=...)| \\
or \verb|Language(grammar_file=...)|.

\paragraph{NOTE (2018-06-16):} In order for this to work, the start rule of the grammer \textbf{must} be called \verb|start|. 
Although there are several ways tot specify another start rule in Tatsu, none of these seemed to work. 
This might be a bug in Tatsu. 
However, the workaround is easy, just declare a rule \verb|"start = c_program $"| and change to \verb|"c_program = c_combinator_list"| (without the \$) and you're done.

\paragraph{2018-04-17T23:00}

The Core grammer had to be modified in three dfferent ways to make it compatible with tatsu PEG requirements:

\begin{enumerate}

\item The Lester grammar as given in Lester is ambiguous, as the semicolon in the rule 
\verb|alts -> alt1 ';' ... ';' altn|
is indistinguishable from the one between super combinators:\\
\verb|program -> sc1 ';' ... ';' scn|.

This anbiguity can besolved in various ways, I have chosen to modify the lexical structure by replacing the semicolon between alterbatives by the Dijkstra block \verb|[]|.

\item The Lester grammar is incomplete as it only specifies the precedence of infix operators in a table. This precedence structure has to be expanded into the grammar rules, giving rise to no less than 8 additional rules.

\item Tatsu specifies semantic functions on a per-rule base, and not on a per-alternative base. In order to define semantics at an appropriate level, many rules of the form
\verb|lhs -> rhs-expansion-1 | rhs-expansion-2 | rhs-expansion-3|, 
where each expansion is a sequence of (non)terminals must be rewritten into the following form:

\begin{Verbatim}
lhs -> nonterminal-1 | nonterminal-2 | nonterminal-3;
nonterminal-1 = rhs-expansion-1;
nonterminal-2 = rhs-expansion-2;
nonterminal-3 = rhs-expansion-3;|
\end{Verbatim}
where each nonterminal-i is a new nonterminal, not occuring in the original grammar.
\end{enumerate}

\paragraph{2019-01-27T23:09}
Migrated two txt files into a single tex file.

\paragraph{NOTE:}
I am now using git gui in stead of smartgit.
My login credentials for github are stored in \verb|~/.git-credentials|.
git gui will pick these up after executing \verb|git config credential.helper store| in the workspace directory.

\paragraph{2019-02-04T12:15}
Lester \& Peyton Jones describe a rather complicated pretty-printer.
They justify this by a time-complexity argument.
In many languages (Miranda, Haskell and Prospero among them) concatenation of strings is quadratic in the size of the left argument.
As they prefer linear time-complexity (for good reasons!) they introduce this complex algorithm.
However, reducing time-complexity comes at a price: the algorithm consumes a large amount of space (although still linear in the size of the program).

Pretty-printing is a two-phase process: in the first phase the parse tree is transformed into a huge structure of IAppend, IString and INil nodes (and many others, as we will see later on).
In the second phase, this structure is quite cleverly flattened to a string, indeed in linear time.

However, this approach needs some critical evaluation:

\begin{enumerate}
\item The first phase builds a structure whic I will call the print graph, whose size is about an order of magnitude larger than the original parse tree.

\item Quadratic time-complexity causes unacceptable execution times for larger programs.
However, for small to mid-size programs on a fast machine, the difference between a fast quadratic algortihm and a slow linear one may be hardly noticable.

\item Research shows that Python optimizes concatenation such that under the right circumstances this operation becomes linear.

\item Maintenance of the two-phase algorithm is difficult, because changes have to be made at two places.
The first phase is build into the nodes of the parse tree and the second phase is implemented as a method in the pretty printer module.

\end{enumerate}

Given these facts, there is no longer a justification for the complicated LPJ algorithm.
However, implementing it was the only way to really understand it end without implementation the above criticism had never been exposed.

\end{document}
